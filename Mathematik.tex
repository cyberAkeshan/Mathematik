\documentclass{article}

\usepackage{graphicx}
\usepackage{hyperref}
\usepackage{amsmath}
\usepackage{amsfonts}
\usepackage{lmodern}
\usepackage[latin1, utf8]{inputenc}% erm\"oglich die direkte Eingabe der Umlaute 
\usepackage[T1]{fontenc} % das Trennen der Umlaute
\usepackage{ngerman} % hiermit werden deutsche Bezeichnungen genutzt und 
\usepackage[left=3cm,right=3cm,top=1cm,bottom=2cm,includeheadfoot]{geometry}
\usepackage{fancyhdr}
\usepackage{pgf-pie} 
\usepackage{setspace} 

%Wie ändert die Corona-Krise das Kommunikationsverhalten
\title{Mathematik 1 mal 1}
\author{Akeshan Kunarajah \\ Fakultät 8 \\
Heidelsheim Lehranstalt}

\date{\today}


\begin{document}
\maketitle
\tableofcontents
\thispagestyle{empty}


\newpage
\setcounter{page}{1}
\section{Grundlegende Mathematik}
\subsection{Potenzregeln}

\begin{align}
a^{0}=1
\end{align}

\begin{align}
a^{1}=a
\end{align}

\begin{align}
a^{m} \cdot a^{n} = a^{m+n}
\end{align}

\begin{align}
(a^{m})^{n} = a^{m \cdot n}
\end{align}

\begin{align}
a^{n} \cdot b^{n} = (ab)^{n}
\end{align}

\begin{align}
a^{-n} = \frac{1}{a^{n}}
\end{align}

\begin{align}
\frac{a^{n}}{a^{m}} = a^{n-m}
\end{align}

\begin{align}
a^{\frac{1}{n}} = \sqrt[n]{a}
\end{align}

\begin{align}
a^{\dfrac{m}{n}} = \sqrt[n]{a^{m}}
\end{align}

\subsubsection{Quadratische Gleichungen}
\begin{align}
x^{2} + px + p = 0
\\
\longrightarrow x_{1,2} = -\frac{p}{2} \pm \sqrt{(\frac{p}{q})^{2} - q}
\end{align}

\begin{align}
ax^{2} + bx + c = 0
\\
\longrightarrow x_{1,2}=\frac{-b \pm \sqrt{b^{2}-4 \cdot a \cdot c}}{2 \cdot a}
\end{align}

\subsubsection{Binomische Formeln}
\begin{align}
(a+b)^{2} = a^{2} + 2ab + b^{2}
\end{align}

\begin{align}
(a-b)^{2} = a^{2} -2ab + b^{2}
\end{align}

\begin{align}
(a+b)(a-b) = a^{2}-b^{2}
\end{align}

\newpage
\section{Vertiefungskurs Mathematik}

\subsection{Konvergenz}
Sei $(a_{n})$ eine Folge. Diese Folge ist genau dann konvergent, wenn sie einen Grenzwert $a$ besitzt, sodass für alle $\epsilon > 0$ ein $n_{0}\in\mathbb{N}$ existiert mit $|a_{n}-a|<\epsilon$ \, $\forall n\geq n_{0}$.

\begin{align}
\lim_{n \to \infty} a_{n} = a
\end{align}

Beispiel 1:

\begin{align}
\lim_{n \to \infty} \frac{1}{n} = 0 \hspace{10pt} \rightarrow Nullfolge
\end{align}

Beispiel 2:

\begin{align}
\lim_{n \to \infty} \frac{9n^{4}-3n^{2}}{3n^{4}-4n^{3}+2n} =  \frac{n^{4}(9-\frac{3}{n^{2}})}{n^{^{4}}(3-\frac{4}{n}+\frac{2}{n^{3}})} =  \frac{(9-\frac{3}{n^{2}})}{(3-\frac{4}{n}+\frac{2}{n^{3}})} = \frac{9}{3} = 3 
\end{align}

\subsection{Beschränktheit}
Eine Folge $(a_{n})$ ist dann beschränkt, wenn es zwei Zahlen $s$ und $S$ gibt, so dass jedes Glied der Folge $(a_{n})$ größer s und kleiner S ist. \\Es gilt also:

\begin{align}
s\leq a_{n} \leq S \hspace{8pt} \forall n \in \mathbb{N}
\end{align}
Somit ist $(a_{n})$ beschränkt.

\end{document}